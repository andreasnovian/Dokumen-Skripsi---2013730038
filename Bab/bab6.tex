\chapter{Kesimpulan dan Saran}
\label{chap:kesimpulanDanSaran}

Pada bab ini dijelaskan mengenai beberapa kesimpulan yang dapat diambil dari penelitian ini dan diberikan beberapa saran yang dapat dilakukan untuk memperbaiki dan mengembangkan penelitian ini selanjutnya.

\section{Kesimpulan}
\label{sec:kesimpulan}

Beberapa kesimpulan yang dapat diambil dari penelitian ini adalah sebagai berikut.

\begin{enumerate}
	\item Aturan morfologi dalam bahasa Indonesia terdiri atas beberapa hal, yaitu:
	\begin{itemize}
		\item Proses morfologi dalam bahasa Indonesia terdiri dari beberapa jenis, yaitu proses pembubuhan afiks dalam proses afiksasi, pengulangan dalam proses reduplikasi, dan penggabungan dalam proses komposisi
		\item Terdapat beberapa aturan dalam proses morfologi bahasa Indonesia, yaitu morfofonemik yang mengatur perubahan bunyi atau perubahan fonem ketika dilakukan proses morfologi dan morfotaktik yang mengatur suatu morfem boleh digabungkan dengan morfem apa saja
	\end{itemize}
	
	\item Leksikon yang digunakan dalam perangkat lunak memiliki spesifikasi sebagai berikut:
	\begin{itemize}
		\item Leksikon menyimpan seluruh kata dasar beserta kata turunan untuk setiap kata dasar yang valid dalam bahasa Indonesia dengan menggunakan struktur data trie dan diimplementasikan dalam perangkat lunak dengan objek dari kelas HashMap yang dimiliki oleh bahasa pemrograman Java
		\item Perangkat lunak lexicon yang dibuat dapat menyimpan kata turunan yang merupakan hasil dari proses morfologi berupa afiksasi, reduplikasi, atau komposisi
	\end{itemize}
	
	\item Implementasi dari aturan morfologi bahasa Indonesia ke dalam perangkat lunak adalah sebagai berikut:
	\begin{itemize}
		\item Implementasi dari aturan morfofonemik dilakukan dalam perangkat lunak morphological parser dengan menghasilkan semua kemungkinan hasil parsing berdasarkan perubahan fonem yang valid dari kata masukan yang diproses 
		\item Implementasi dari aturan morfotaktik dilakukan dalam perangkat lunak morphological parser dengan melakukan validasi hasil parsing melalui perangkat lunak lexicon
	\end{itemize}
	
	\item Performansi dari perangkat lunak yang dihasilkan adalah sebagai berikut:
	\begin{itemize}
		\item Perangkat lunak morphological parser yang dibuat dapat melakukan proses parsing dalam waktu yang masuk akal, yaitu kurang dari 100 milidetik untuk teks yang terdiri dari 100 kata
		\item Perangkat lunak morphological parser yang dibuat dapat menghasilkan semua kemungkinan hasil parsing yang valid dari sebuah kata
		\item Perangkat lunak lexicon yang dibuat dapat melakukan create, update, dan delete untuk entri kata dasar dan kata turunan yang disimpan dalam leksikon
		\item Operasi create, update, dan delete pada perangkat lunak lexicon membutuhkan waktu sangat singkat, yaitu di bawah 10 milidetik untuk setiap operasi yang dilakukan
	\end{itemize}
	
\end{enumerate}


\section{Saran}
\label{sec:saran}

Beberapa saran yang dapat dilakukan untuk memperbaiki dan mengembangkan penelitian ini selanjutnya adalah sebagai berikut.

\begin{itemize}
	\item Melengkapi seluruh kata turunan untuk setiap kata dasar dalam leksikon supaya semua kata dalam bahasa Indonesia dapat diproses dan divalidasi dengan tepat.
	\item Menambahkan kelas kata seperti kata benda, kata kerja, kata sifat, dan lain-lain untuk setiap kata dasar dan kata turunan yang disimpan dalam leksikon supaya dapat digunakan dalam proses lebih lanjut dalam pengolahan bahasa alami.
\end{itemize}