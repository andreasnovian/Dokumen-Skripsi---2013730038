\chapter{Kode Program}
\label{lamp:B}

%terdapat 2 cara untuk memasukkan kode program
% 1. menggunakan perintah \lstinputlisting (kode program ditempatkan di folder yang sama dengan file ini)
% 2. menggunakan environment lstlisting (kode program dituliskan di dalam file ini)
% Perhatikan contoh yang diberikan!!
%
% untuk keduanya, ada parameter yang harus diisi:
% - language: bahasa dari kode program (pilihan: Java, C, C++, PHP, Matlab, C#, HTML, R, Python, SQL, dll)
% - caption: nama file dari kode program yang akan ditampilkan di dokumen akhir
%
% Perhatian: Abaikan warning tentang textasteriskcentered!!
%

\lstinputlisting[language=Java, caption=Parser.java]{./Lampiran/src/Model/Parser.java} 
\lstinputlisting[language=Java, caption=Lexicon.java]{./Lampiran/src/Lexicon/Model/Lexicon.java} 
\lstinputlisting[language=Java, caption=Node.java]{./Lampiran/src/Lexicon/Model/Node.java}
\lstinputlisting[language=Java, caption=Combiner.java]{./Lampiran/src/Lexicon/Model/Combiner.java}
